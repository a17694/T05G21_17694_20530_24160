\chapter{Conclusão}
Este trabalho serviu para compreender como se encontra estruturada a companhia aérea Air Astana, no que diz respeito às suas rotas de voo. Usou-se métodos matemáticos para várias determinações, como por exemplo, os fechos transitivos ajudam-nos a perceber que certos destinos só são alcançáveis a partir de outros, se fizermos várias escalas em 2 ou 3 aeroportos. Na realidade, só acontece caso o cliente seja muito fiel à companhia em questão.\\
\indent Não será exagero de todo pensar que se o conhecimento da teoria de grafos fosse de cultura geral e aplicado no dia a dia pelo cidadão comum, ponderaríamos bem mais sobre certas decisões a tomar, a bem da eficiência de gastos em viagens de carro, entre outros.\\
\indent A nível empresarial, parece claro que o conceito da teoria de grafos é muito aplicado. Podemos ver tal facto pelo próprio mapa interativo de onde se retirou a informação das rotas aéreas, que serviram de base para este trabalho. Sabemos que algoritmos de grafos estão presentes em sites traduzidos em interfaces que o público interage para ver as rotas desejadas. A gestão de rotas também é feita com base nestes conceitos, pois caso assim não fosse, haveria muitas mais rotas de voo desnecessárias que não trazem lucro à companhia.\\
\indent Admitimos que deixamos de fora deste trabalho alguns aspetos que complementariam mais o tema, mas para nos focarmos no essencial da questão, foi a nosso ver a melhor solução a adotar.\\
\indent Após este trabalho, e feita uma análise mais aprofundada deste tema, podemos dizer que a teoria de grafos é interessante e que deveríamos aplicar este assunto no nosso dia a dia.

%% Altera o titulo de bibliografia
\renewcommand\bibname{Referências}
\begin{thebibliography}{3}
    \bibitem{link1}
    \href{https://airastana.com/global/en-us}{Airastana}
    \bibitem{link2}
    \href{https://www.flightconnections.com/}{Flightconnections.com}
    \bibitem{link3}
    \href{https://www.flightconnections.com/route-map-air-astana-kc}{www.flightconnections.com/route-map-air-astana-kc}
\end{thebibliography}