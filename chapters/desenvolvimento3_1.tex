\section{ALGUNS SUBCAPÍTULOS DA TEORIA DE GRAFOS}
\subsection{Grafos simples e multigrafo}
- \underline{Grafo simples}, onde entre dois vértices estão conectados por apenas uma aresta. Isto é válido para 
todos os vértices do grafo.\\
\figure
\\
- \underline{Multigrafo}, no caso de terem várias arestas a conectar os mesmos dois vértices, ou no caso de um 
vértice possuir um lacete, ou seja, uma “aresta” a ligar esse mesmo vértice nas duas extremidades.\\
label 
\\
\begin{flushright}
    $($Nas companhias aéreas os passageiros detestam lacetes.\\O avião descola do aeroporto e pouco depois\\aterra no mesmo novamente!$)$
\end{flushright}

\subsection{Digrafos ou grafos orientados}
São grafos que apresentam na sua totalidade arestas orientadas, mais conhecidas como arcos.
%(inserir figura 5)
\\
\subsection{Grafo Parcial e Subgrafo gerado de um dado grafo}
